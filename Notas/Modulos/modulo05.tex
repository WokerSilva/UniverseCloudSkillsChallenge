\section{Introducción a GitHub}

%\\\\\\\\\\\\\\\\\\\\\\\\\\\\\\\\\\\\\\\\\\\\\
%\\\\\\\\\\\\\\\\\\\\\\\\\\\\\\\\\\\\\\\\\\\
%\\\\\\\\\\\\\\\\\\\\\\\\\\\\\\\\\\\\\\\\\
\subsection*{Introducción}
%\\\\\\\\\\\\\\\\\\\\\\\\\\\\\\\\\\\\\\\\\
%\\\\\\\\\\\\\\\\\\\\\\\\\\\\\\\\\\\\\\\\\\\
%\\\\\\\\\\\\\\\\\\\\\\\\\\\\\\\\\\\\\\\\\\\\\

GitHub proporciona una plataforma para desarrolladores con tecnología de inteligencia artificial
para compilar, escalar y entregar software seguro. Tanto si planea nuevas características, corregir
errores o colaborar en los cambios, GitHub es donde más de 100 millones de desarrolladores de todo el 
mundo se unen para crear cosas y mejorarlas aún más.

En este módulo, aprenderá los conceptos básicos de GitHub y comprenderá mejor sus características 
fundamentales con un ejercicio práctico en un repositorio de GitHub.

%\\\\\\\\\\\\\\\\\\\\\\\\\\\\\\\\\\\\\\\\\\\\\
%\\\\\\\\\\\\\\\\\\\\\\\\\\\\\\\\\\\\\\\\\\\
%\\\\\\\\\\\\\\\\\\\\\\\\\\\\\\\\\\\\\\\\\
\subsection*{¿Qué es GitHub?}
%\\\\\\\\\\\\\\\\\\\\\\\\\\\\\\\\\\\\\\\\\
%\\\\\\\\\\\\\\\\\\\\\\\\\\\\\\\\\\\\\\\\\\\
%\\\\\\\\\\\\\\\\\\\\\\\\\\\\\\\\\\\\\\\\\\\\\


GitHub es una plataforma basada en la nube que usa Git, un sistema de control de versiones 
distribuido, en su núcleo. La plataforma GitHub simplifica el proceso de colaborar en proyectos
y proporciona un sitio web, herramientas de línea de comandos y un flujo global que permite a
los desarrolladores y usuarios trabajar juntos.

Como hemos aprendido anteriormente, GitHub proporciona una plataforma para desarrolladores con 
tecnología de inteligencia para crear, escalar y ofrecer software seguro. Vamos a desglosar cada 
uno de los pilares básicos de la plataforma GitHub Enterprise, inteligencia artificial, colaboración, 
productividad, seguridad y escala.

%\\\\\\\\\\\\\\\\\\\\\\\\\\\\\\\\\\\\\\\\\\\\\
%\\\\\\\\\\\\\\\\\\\\\\\\\\\\\\\\\\\\\\\\\\\
%\\\\\\\\\\\\\\\\\\\\\\\\\\\\\\\\\\\\\\\\\
\subsubsection*{INTELIGENCIA ARTIFICIAL}
%\\\\\\\\\\\\\\\\\\\\\\\\\\\\\\\\\\\\\\\\\
%\\\\\\\\\\\\\\\\\\\\\\\\\\\\\\\\\\\\\\\\\\\
%\\\\\\\\\\\\\\\\\\\\\\\\\\\\\\\\\\\\\\\\\\\\\


La inteligencia artificial generativa está transformando drásticamente el desarrollo de software a medida que hablamos.

La plataforma GitHub Enterprise mejora la colaboración a través de solicitudes de incorporación de
cambios y problemas con tecnología de inteligencia artificial, la productividad a través de Copiloto
y la seguridad mediante la automatización de las comprobaciones de seguridad más rápido.

\subsubsection*{Colaboración}

En esencia, colaboración de todo lo que hace GitHub. Sabemos que la colaboración ineficaz da
como resultado tiempo y dinero desperdiciados. Lo contrarrestamos con un conjunto de herramientas 
sin fisuras que permiten colaborar sin esfuerzo.

Los repositorios, las incidencias, las solicitudes de incorporación de cambios y otras 
herramientas ayudan a los desarrolladores, administradores de proyectos, líderes de operaciones y 
otros usuarios de la misma empresa a trabajar más rápido, reducir los tiempos de aprobación y enviar más rápidamente.

\subsubsection*{Productividad}

La productividad se acelera con la automatización que proporciona la plataforma GitHub
Enterprise. Con las herramientas de CI/CD integradas directamente en el flujo de trabajo,
la plataforma ofrece a los usuarios la capacidad de establecer tareas y olvidarlas, cuidar de 
la administración rutinaria y acelerar el trabajo diario. Esto proporciona a los desarrolladores 
más tiempo para centrarse en lo que más importa: crear soluciones innovadoras.

\subsubsection*{Seguridad}

GitHub se centra en integrar la seguridad directamente en el proceso de desarrollo desde el 
principio. La plataforma GitHub Enterprise incluye características de seguridad nativas y de 
primera entidad que minimizan el riesgo de seguridad con una solución de seguridad integrada. 
Además, el código permanece privado dentro de su organización y, al mismo tiempo, puede 
aprovechar las ventajas de la información general de seguridad y Dependabot.

GitHub ha seguido realizando inversiones para asegurarse de que nuestras características estén 
listas para la empresa. Estamos respaldados por Microsoft, que confía en sectores altamente 
regulados y cumplen los requisitos de cumplimiento globalmente.

\subsubsection*{Escala}

GitHub es la comunidad de desarrolladores más grande de su tipo. Con datos en tiempo real en
más de 100 000 desarrolladores, más de 330 000 repositorios e innumerables implementaciones, 
hemos podido comprender las necesidades cambiantes de los desarrolladores y realizar cambios 
en nuestro producto para adaptarnos a ellas.

Esto se ha traducido en una escala increíble que no tiene parangón ni comparación con ninguna 
otra empresa del planeta. Cada día obtenemos más información de esta impresionante comunidad y 
hacemos evolucionar la plataforma para satisfacer sus necesidades.

En esencia, la plataforma GitHub Enterprise se centra en la experiencia del desarrollador: tiene 
la escala necesaria para ofrecer perspectivas que cambian el sector, capacidades de colaboración 
para una eficiencia transformadora, las herramientas para aumentar la productividad, seguridad 
en cada paso y la inteligencia artificial para impulsarlo todo a nuevas cotas en una 
única plataforma integrada.

\subsection*{Introducción a Repositorios}

¿Qué es un repositorio?
Creación de un repositorio
¿Qué son los gists?
¿Qué son las wikis?


\subsection*{Componentes del flujo de GitHub}

Ramas
Confirmaciones
    Sin modificar: se realiza un seguimiento del archivo, pero no se ha modificado desde la última confirmación.
    Modificado: el archivo se ha cambiado desde la última confirmación, pero estos cambios aún no están almacenados provisionalmente para la siguiente confirmación.
    Almacenado provisionalmente: el archivo se ha modificado y los cambios se han agregado al área de almacenamiento provisional (también conocida como índice). Estos cambios están listos para confirmarse.
    Confirmado: el archivo se encuentra en la base de datos del repositorio. Representa la versión confirmada más reciente del archivo.


Qué son las solicitudes de incorporación de cambios?
    

El flujo de GitHub

l primer paso del flujo de GitHub consiste en crear una rama para que los cambios, características y correcciones que cree no afecten a la rama principal.
El segundo paso es realizar los cambios. Se recomienda implementar cambios en la rama de características antes de combinarlos en la rama principal. De esta forma, se tiene la seguridad de que los cambios son válidos en un entorno de producción.
El tercer paso consiste en crear una solicitud de incorporación de cambios para pedir comentarios a los colaboradores. La revisión de solicitude sde cambios es tan valiosa que algunos repositorios requieren una revisión aprobatoria antes de que estas se puedan fusionar.
A continuación, el cuarto paso consiste en revisar e implementar los comentarios de los colaboradores.
Una vez que se sienta a gusto con los cambios, el quinto paso es aprobar la solicitud de incorporación de cambios y combinarla en la rama principal.
El sexto y último paso es eliminar la rama. Al eliminar la rama se indica que el trabajo en la rama se ha completado y se evita que usted u otros usuarios empleen accidentalmente ramas antiguas.


\subsection*{GitHub es una plataforma colaborativa}

Incidencias
Debates
Habilitación de un debate en el repositorio


\subsection*{Qué son las páginas de GitHub}
Para finalizar nuestro recorrido por GitHub, analicemos las páginas de GitHub.

Puede usar páginas de GitHub para publicitar y hospedar un sitio web sobre usted, su organización o su proyecto directamente desde un repositorio de GitHub.com.

GitHub Pages es un servicio de hospedaje de sitios estáticos que toma archivos HTML, CSS y JavaScript directamente desde un repositorio de GitHub. Opcionalmente, puede ejecutar los archivos a través de un proceso de compilación y publicar un sitio web.

Simplemente edite e introduzca los cambios y ya está, su proyecto estará disponible para el público de una manera visualmente organizada.

A continuación, le guiaremos por un ejercicio para empezar a trabajar con GitHub.

Podrá:

Creación de un nuevo repositorio
Creación de una rama
Confirmar un archivo
Apertura de una solicitud de incorporación de cambios
Y combinar una solicitud de cambios