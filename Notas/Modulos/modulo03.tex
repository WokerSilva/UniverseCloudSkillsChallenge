\section{Colaboración con Git}

%/////////////////////////////////////////////
%///////////////////////////////////////////
%/////////////////////////////////////////
\subsection*{Clonacion de un repositorio}
%\\\\\\\\\\\\\\\\\\\\\\\\\\\\\\\\\\\\\\\\\
%\\\\\\\\\\\\\\\\\\\\\\\\\\\\\\\\\\\\\\\\\\\
%\\\\\\\\\\\\\\\\\\\\\\\\\\\\\\\\\\\\\\\\\\\\\


En Git, un repositorio se copia al clonarlo mediante el comando \myBox{sun}{git clone}. Puede
clonar un repositorio independientemente de dónde esté almacenado, siempre que tenga una
dirección URL o una ruta de acceso a la que apuntar. En Unix y Linux, la operación de clonación
usa vínculos físicos, así que es rápida y usa un espacio mínimo, ya que solo hay que copiar las
entradas de directorio, no los archivos.

%/////////////////////////////////////////////
%///////////////////////////////////////////
%/////////////////////////////////////////
\subsection*{Repositorio remoto}
%\\\\\\\\\\\\\\\\\\\\\\\\\\\\\\\\\\\\\\\\\
%\\\\\\\\\\\\\\\\\\\\\\\\\\\\\\\\\\\\\\\\\\\
%\\\\\\\\\\\\\\\\\\\\\\\\\\\\\\\\\\\\\\\\\\\\\

Cuando se clona repositorio en Git, establece una referencia al repositorio original llamado
\textit{origin}. Esto facilita la incorporación y envío de cambios. El comando utilizado es
\myBox{sun}{git pull}, copia las confirmaciones y objetos nuevos del repositorio remoto al local.
A diferencia de otros métodos como scp o Rsync, Git solo examina las confirmaciones, no todos
los archivos.

Git guarda la lista de confirmaciones obtenidas, y al usar \myBox{sun}{git pull}, solicita
al repositorio remoto enviar solo los cambios, incluyendo nuevas confirmaciones y objetos.
Estos se agrupan en un archivo llamado paquete y se envían en un lote. Luego, Git actualiza
el árbol de trabajo al desempaquetar y combinar estos objetos con las confirmaciones locales.

Es importante destacar que Git solo incorpora o envía cambios cuando el usuario lo indica, a
diferencia de sistemas como Dropbox que dependen del sistema operativo para notificar cambios
en la carpeta y consultan al servidor sobre posibles modificaciones de otros usuarios.

%/////////////////////////////////////////////
%///////////////////////////////////////////
%/////////////////////////////////////////
\subsection*{Creación de solicitudes de incorporación de cambios}
%\\\\\\\\\\\\\\\\\\\\\\\\\\\\\\\\\\\\\\\\\
%\\\\\\\\\\\\\\\\\\\\\\\\\\\\\\\\\\\\\\\\\\\
%\\\\\\\\\\\\\\\\\\\\\\\\\\\\\\\\\\\\\\\\\\\\\

\myBox{sun}{git request-pull <inicio> <fin> <repositorio remoto>}

Este comando en Git se utiliza para generar y enviar solicitudes formales de incorporación
de cambios. Al especificar el rango de cambios entre dos puntos (inicio y fin) y el repositorio
remoto destinatario, se crea una solicitud que incluye detalles sobre las modificaciones
propuestas. Esto facilita el proceso de revisión y aceptación por parte del propietario del
repositorio remoto, permitiendo una integración eficiente de contribuciones externas al proyecto.

Supongamos que he estado trabajando en una nueva función en mi rama local llamada \textit{nueva-funcion}
y quiero solicitar la incorporación de estos cambios al repositorio remoto llamado \textit{repositorio-remoto}.
Utilizaría el comando git request-pull de la siguiente manera:


\begin{itemize}
    \item[] git request-pull
    \begin{itemize}
        \item[] origin/nueva-funcion
        \item[] https://github.com/usuario/repositorio-remoto.git
    \end{itemize}
\end{itemize}
  

\begin{itemize}
    \item[] origin/nueva-funcion: es la rama que contiene los cambios que deseo incorporar.
    \item[] https://github.com/usuario/repositorio-remoto.git: es la URL del repositorio.
\end{itemize}


Este comando generaría un resumen de los cambios entre el estado actual de mi rama \textit{nueva-funcion}
y su punto de inicio. Luego, puedo enviar este resumen al propietario del repositorio remoto para
que revisen y consideren la incorporación de mis cambios al proyecto principal.




%/////////////////////////////////////////////
%///////////////////////////////////////////
%/////////////////////////////////////////
\subsection*{}
%\\\\\\\\\\\\\\\\\\\\\\\\\\\\\\\\\\\\\\\\\
%\\\\\\\\\\\\\\\\\\\\\\\\\\\\\\\\\\\\\\\\\\\
%\\\\\\\\\\\\\\\\\\\\\\\\\\\\\\\\\\\\\\\\\\\\\

