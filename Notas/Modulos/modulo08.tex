\section{Proyecto de desafío}

\titles{ Crear una aplicación de consola de minijuegos con GitHub Copilot}

\subsection{Introducción}

La creación de un minijuego puede ayudarle a practicar sus habilidades de programación y mejorar 
su capacidad para crear aplicaciones de consola en Python.

En este módulo, desarrollará el minijuego clásico de piedra, papel, tijeras con la ayuda de 
GitHub Codespaces y GitHub Copilot. Es decir, no es necesario preocuparse por configurar el 
entorno de desarrollo, por lo que puede centrarse en el desarrollo de aplicaciones mientras 
se basa en un código asistente.


\subsection{Ejercicio: Crear la lógica del juego}

Especificación

\begin{itemize}
    \item Reglas del juego:
    \begin{itemize}
        \item La piedra gana a las tijeras (las rompe).
        \item Las tijeras han ganado al papel (lo cortan).
        \item El papel gana a la piedra (la envuelve).
        \item El minijuego es multijugador y el equipo juega el papel del oponente 
                y elige un elemento aleatorio de la lista de elementos
    \end{itemize}
    \item Interacción con el jugador:
    \begin{itemize}
        \item La consola se usa para interactuar con el jugador.
        \item El jugador puede elegir una de las tres opciones: roca, papel o tijeras.
        \item El jugador puede elegir si vuelve a jugar.
        \item Se debe advertir al jugador si introduce una opción no válida.
        \item El jugador ve su puntuación al final del juego.
    \end{itemize}
    \item Validación de la entrada del usuario:
     \begin{itemize}
        \item En cada ronda, el jugador debe entrar en una de las opciones de la lista 
                y ser informado de si ganó, perdió o empató con el oponente.
        \item El minijuego debe controlar las entradas del usuario, colocarlas en 
                minúsculas e informar al usuario si la opción no es válida.
        \item Al final de cada ronda, el jugador debe responder si quiere jugar de nuevo o no.
     \end{itemize}
\end{itemize}

\begin{verbatim}
Vamos a crear el juego de piedra papel o tijeras en python. La computadora es 
le jugador 1 y el usuario es el jugador 2
\end{verbatim}

\begin{verbatim}
    Vamos a crear el juego de piedra papel o tijeras en python. La computadora es 
    le jugador 1 y el usuario es el jugador 2
\end{verbatim}