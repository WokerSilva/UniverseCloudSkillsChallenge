\section{Introducción a Git}

%/////////////////////////////////////////////
%///////////////////////////////////////////
%/////////////////////////////////////////
\subsection{Importancia del Control de Versiones}
%\\\\\\\\\\\\\\\\\\\\\\\\\\\\\\\\\\\\\\\\\
%\\\\\\\\\\\\\\\\\\\\\\\\\\\\\\\\\\\\\\\\\\\
%\\\\\\\\\\\\\\\\\\\\\\\\\\\\\\\\\\\\\\\\\\\\\

El control de versiones es esencial en el desarrollo de software, ya que permite gestionar los cambios en el código fuente
y la colaboración entre desarrolladores de manera eficiente y segura. Git es un VCS de código abierto rápido, versátil, muy
escalable y gratuito. Su autor principal es Linux Torvalds, creador de Linux. \\

Imaginemos que estamos escribiendo un reporte de practica con nuestro equipo, seguramente a muchos nos paso que todos nos
conectábamos al mismo tiempo escribiendo sobre el documento haciendo un mezcla en todo el texto, agregando referencias o 
imágenes, incluso hasta borrando cosas de los demás. Usando control de versiones, cada modificación se registra,
permitiendo ver quién hizo qué y cuándo.\\

%/////////////////////////////////////////////
%/////////////////////////////////////////////
\titles{¿Qué ofrece un control de versiones?}
%\\\\\\\\\\\\\\\\\\\\\\\\\\\\\\\\\\\\\\\\\\\\\
%\\\\\\\\\\\\\\\\\\\\\\\\\\\\\\\\\\\\\\\\\\\\\

\begin{itemize}
    \item \textbf{Historial de Cambios:} Un controlador de versiones mantiene un registro detallado de todos los cambios
                                         realizados en el código a lo largo del tiempo, lo que permite rastrear quién hizo qué y cuándo.                                              
    \item \textbf{Ramificación y Fusiones:} Permite la creación de ramas \textit{(branching)} para trabajar en nuevas características
                                            o correcciones sin afectar la versión principal. Posteriormente, se pueden fusionar 
                                            \textit{(mergear)} estas ramas de manera controlada.
    \item \textbf{Recuperación de Versiones Anteriores:} Si surge un problema en una versión de software, es posible volver
                                                         a una versión anterior para solucionar el problema rápidamente.
    \item \textbf{Copias de Seguridad:} Los controladores de versiones actúan como una copia de seguridad automática del código fuente.
                                        En el caso de Git, cada clon del repositorio es una copia completa del historial,
                                        lo que garantiza la seguridad y la disponibilidad de los datos.
    \item \textbf{Revisiones y Pruebas Colaborativas:} Los miembros del equipo pueden realizar revisiones de código de manera conjunta,
                                                       identificar problemas y realizar pruebas en paralelo, acelerando el proceso de desarrollo.
\end{itemize}

%/////////////////////////////////////////////
%/////////////////////////////////////////////
\titles{Control de versiones distribuido}
%\\\\\\\\\\\\\\\\\\\\\\\\\\\\\\\\\\\\\\\\\\\\\
%\\\\\\\\\\\\\\\\\\\\\\\\\\\\\\\\\\\\\\\\\\\\\

Los sistemas de control de versiones centralizados, como \texttt{CVS, SVN} y \texttt{Perforce},
almacenan el historial de un proyecto en un solo servidor. Esto puede ser un problema si el
servidor falla o está inaccesible. Git es un sistema de control de versiones distribuido, 
lo que significa que el historial de un proyecto se almacena en el cliente y en el servidor.
Esto hace que Git sea más robusto y fiable que los sistemas centralizados.

%/////////////////////////////////////////////
%///////////////////////////////////////////
%/////////////////////////////////////////
\section{Terminología de Git}
%\\\\\\\\\\\\\\\\\\\\\\\\\\\\\\\\\\\\\\\\\
%\\\\\\\\\\\\\\\\\\\\\\\\\\\\\\\\\\\\\\\\\\\
%\\\\\\\\\\\\\\\\\\\\\\\\\\\\\\\\\\\\\\\\\\\\\


\subsubsection*{Árbol de trabajo}

Conjunto de directorios y archivos anidados que contienen el proyecto en el que se trabaja.

\subsubsection*{Repositorio (repo):}
Un repositorio, también conocido como repo, es un directorio situado en el nivel superior de un árbol de trabajo en Git,
donde se almacena todo el historial y los metadatos de un proyecto. Puede haber repositorios vacíos, que no forman parte
de un árbol de trabajo y se utilizan para compartir o realizar copias de seguridad. Un repositorio vacío generalmente es
un directorio con un nombre que termina en \myBox{sun}{.git}, por ejemplo, project.git.

\subsubsection*{Inicializar un Repo:} 
Es el proceso de crear una nueva copia de trabajo de Git y asociarla con un remoto. Este proceso se puede realizar mediante
el comando \myBox{sun}{git init}.

\subsubsection*{Hash:}
Un hash es un número generado por una función hash que representa el contenido de un archivo u otro objeto como un
número de dígitos fijo. En Git, se utilizan hashes de 160 bits de longitud. La ventaja de los códigos hash en Git es
que permiten verificar si un archivo ha cambiado mediante la comparación de su hash actual con el hash anterior. Si
el contenido del archivo no cambia, aunque se modifique la marca de fecha y hora, el hash seguirá siendo el mismo.

\subsubsection*{Objeto:}
Un repositorio de Git contiene cuatro tipos de objetos, cada uno identificado de forma única por un hash SHA-1. 
Un objeto blob contiene un archivo normal, un objeto árbol representa un directorio y contiene nombres, valores
hash y permisos, un objeto de confirmación representa una versión específica del árbol de trabajo, y una etiqueta
es un nombre asociado a una confirmación en Git.

\subsubsection*{Confirmación:}
El término \textit{confirmación} puede usarse como un verbo para referirse a la acción de crear un objeto
de confirmación en Git. Esto implica guardar los cambios realizados en un proyecto para que otros usuarios puedan acceder a ellos.

\subsubsection*{Rama:}
Una rama en Git es una serie de confirmaciones vinculadas con un nombre. La confirmación más reciente en una rama
se conoce como el nivel superior de esa rama. La rama predeterminada, que se crea al inicializar un repositorio,
se llama \myBox{sun}{main} y suele tener el nombre \myBox{sun}{master} en Git. La rama actual se identifica como \myBox{sun}{HEAD}.
Las ramas son una característica poderosa de Git que permite a los desarrolladores trabajar de forma independiente o colaborativa
en diferentes ramas y luego fusionar los cambios en la rama predeterminada.

\subsubsection*{Remoto:}
Un remoto en Git es una referencia con nombre a otro repositorio de Git. Al crear un repositorio, Git generalmente 
crea un remoto llamado \myBox{sun}{origin}, que es el remoto predeterminado para las operaciones de envío e incorporación de cambios.

\subsubsection*{Comandos, subcomandos y opciones:}
Las operaciones en Git se realizan mediante comandos, subcomandos y opciones. El comando principal,
como \myBox{sun}{git push} o \myBox{sun}{git pull}, especifica la operación que se desea realizar. Los comandos suelen ir acompañados de 
opciones que se utilizan con guiones (-) o guiones dobles (--). Por ejemplo, \myBox{sun}{git reset --hard} es un comando que
implica un subcomando \myBox{sun}{reset} con la opción \myBox{sun}{--hard}.



%/////////////////////////////////////////////
%/////////////////////////////////////////////
\titles{Git y GitHub}
%\\\\\\\\\\\\\\\\\\\\\\\\\\\\\\\\\\\\\\\\\\\\\
%\\\\\\\\\\\\\\\\\\\\\\\\\\\\\\\\\\\\\\\\\\\\\

Git es un sistema de control de versiones distribuido (DVCS) que varios desarrolladores y otros colaboradores pueden usar
para trabajar en un proyecto. Proporciona una manera de trabajar con una o varias ramas locales y luego insertarlas en un repositorio remoto.\\

GitHub es una plataforma en la nube que usa Git como tecnología principal. Simplifica el proceso de colaboración en
proyectos y proporciona un sitio web, más herramientas de línea de comandos y un flujo integral que los desarrolladores
y usuarios pueden usar para trabajar juntos. GitHub actúa como el repositorio remoto mencionado anteriormente.


%/////////////////////////////////////////////
%///////////////////////////////////////////
%/////////////////////////////////////////
\subsection{Configuración de Git}
%\\\\\\\\\\\\\\\\\\\\\\\\\\\\\\\\\\\\\\\\\
%\\\\\\\\\\\\\\\\\\\\\\\\\\\\\\\\\\\\\\\\\\\
%\\\\\\\\\\\\\\\\\\\\\\\\\\\\\\\\\\\\\\\\\\\\\

\begin{itemize}
    \item Para comprobar que Git está instalado, usar el comando \myBox{sun}{git --version}
    \item Para configurar Git, se define por variables globales: user.name y user.email. Ambas necesarias para hacer confirmaciones.
    \begin{verbatim}
        git config --global user.name <USER_NAME>
        git config --global user.email <USER_EMAIL>
    \end{verbatim}    
    \item Lista detallada de la configuración de GIT \myBox{sun}{git config --list}
\end{itemize}

%/////////////////////////////////////////////
%/////////////////////////////////////////////
\titles{Configuración del repositorio de Git}
%\\\\\\\\\\\\\\\\\\\\\\\\\\\\\\\\\\\\\\\\\\\\\
%\\\\\\\\\\\\\\\\\\\\\\\\\\\\\\\\\\\\\\\\\\\\\

\begin{enumerate}
    \item Cree una carpeta, la cual será el directorio del proyecto y nos movemos a ella
    \begin{verbatim}
        mkdir Cats
        cd Cats
    \end{verbatim}
    \item Inicializar el repositorio 
    \begin{verbatim}
        git init -b main
    \end{verbatim}
    \item Usando el comando 
    \begin{verbatim}
        git status
    \end{verbatim}
    para mostrar el estado del árbol de trabajo
\end{enumerate}
