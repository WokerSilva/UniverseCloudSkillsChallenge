\section{Proyecto de desafío 02}

\titles{Agregar funcionalidades de generación y análisis de imágenes a la aplicación}

\subsection{Introducción}

Computer Vision es un área principal de inteligencia artificial (IA), destinada a procesar y analizar 
la información contenida en orígenes de datos visuales, como imágenes y vídeos, para extraer información 
útil. Esta rama de IA emula las funcionalidades visuales humanas mediante algoritmos avanzados: las 
tecnologías de inteligencia artificial de vanguardia han alcanzado un nivel de precisión comparable a la 
visión humana en muchas tareas, como la clasificación de imágenes, la detección de objetos y la descripción de imágenes.

Saber cómo integrar Computer Vision en sus aplicaciones es fundamental para desarrollar soluciones modernas 
e inteligentes en varios dominios, como el comercio minorista o la seguridad, y para garantizar la accesibilidad a 
los servicios, por ejemplo, proporcionando una descripción de texto de las imágenes que comparte en su sitio web.

\subsection{Azure}

Azure es una plataforma de computación en la nube de Microsoft que ofrece una amplia gama de servicios, desde 
infraestructura como servicio (IaaS) hasta plataforma como servicio (PaaS) y software como servicio (SaaS). 
Azure permite a las empresas de todos los tamaños crear, implementar y administrar aplicaciones en la nube de 
manera rápida y sencilla.

\begin{itemize}
    \item IaaS

    Los servicios IaaS de Azure brindan a las empresas la infraestructura de TI básica que necesitan para ejecutar 
    sus aplicaciones, como servidores, almacenamiento y redes. Azure ofrece una variedad de opciones de IaaS para 
    satisfacer las necesidades de las empresas, desde servidores virtuales hasta centros de datos completos.

    \item PaaS

    Los servicios PaaS de Azure brindan a las empresas un entorno completo para desarrollar, implementar y 
    administrar aplicaciones. Azure ofrece una variedad de servicios PaaS para diferentes tipos de aplicaciones, 
    como aplicaciones web, aplicaciones móviles y aplicaciones de análisis.

    \item SaaS

    Los servicios SaaS de Azure brindan a las empresas software listo para usar que se ejecuta en la nube. 
    Azure ofrece una variedad de servicios SaaS para diferentes necesidades comerciales, como correo 
    electrónico, colaboración, análisis y aprendizaje automático.
\end{itemize}

Algunos de los beneficios de usar Azure incluyen:
\begin{itemize}
    \item Flexibilidad: Azure ofrece una amplia gama de servicios que las empresas pueden usar para satisfacer sus necesidades específicas.
    \item Escalabilidad: Azure es escalable, lo que significa que las empresas pueden aumentar o disminuir sus recursos según sea necesario.
    \item Seguridad: Azure ofrece una variedad de funciones de seguridad para proteger los datos y las aplicaciones de las empresas.
\end{itemize}

\subsection{Azure Static Web App}

Azure Static Web Apps es un servicio de Azure que permite a los desarrolladores crear, implementar y administrar 
aplicaciones web estáticas en la nube. Las aplicaciones web estáticas son aquellas que no requieren un servidor web 
para ejecutarse. En su lugar, se componen de archivos HTML, CSS y JavaScript que se pueden servir directamente 
desde un servidor de archivos.

Para crear una aplicación web estática con Azure Static Web Apps, solo necesitas tener un repositorio de código Git 
con tu aplicación. Luego, puedes usar la CLI de Azure o el portal de Azure para crear una nueva aplicación web estática 
y conectarla a tu repositorio de código.

Una vez que tu aplicación web estática esté creada, Azure la construirá y la implementará en la nube. Tu aplicación 
estará disponible en una URL global que puedes compartir con tus usuarios.


\subsection{Azure OpenAI}


Proporciona acceso a través de la API REST a potentes modelos de lenguaje, como GPT-4, GPT-3.5-Turbo e Embeddings de OpenAI. 
Las series de modelos GPT-4 y GPT-3.5-Turbo ya están disponibles de forma generalizada. Lo genial es que estos modelos se pueden 
adaptar fácilmente a tareas específicas, como generación de contenido, resumen, búsqueda semántica y traducción de 
lenguaje natural a código. Puedes acceder al servicio a través de API REST, el SDK de Python o la interfaz web en 
Azure OpenAI Studio.


\subsection{Seguridad de los servicios de Azure AI}

La seguridad debe considerarse una prioridad máxima en el desarrollo de toda aplicación y, con el crecimiento de las aplicaciones
compatibles con la inteligencia artificial, la seguridad es aún más importante. En este artículo se describen varias 
características de seguridad disponibles para los servicios de Azure AI. Cada característica aborda una responsabilidad 
específica, por lo que se pueden usar varias características en el mismo flujo de trabajo.

\subsection{React}

React es una biblioteca de JavaScript de código abierto que se utiliza para crear interfaces de usuario. Es una de las bibliotecas 
de JavaScript más populares y se utiliza para crear una amplia gama de aplicaciones web, móviles y de escritorio.

React se basa en el concepto de componentes, que son piezas reutilizables de código que pueden combinarse para 
crear interfaces de usuario complejas. Los componentes son fáciles de entender y mantener, lo que hace que React 
sea una buena opción para proyectos de cualquier tamaño.

React se puede utilizar para crear una amplia gama de interfaces de usuario, desde aplicaciones web simples hasta aplicaciones 
web complejas. Algunas de las cosas que se pueden hacer con React incluyen:

\begin{itemize}
    \item Crear aplicaciones web interactivas: React es ideal para crear aplicaciones web interactivas, como aplicaciones de comercio electrónico, aplicaciones de redes sociales y juegos.
    \item Crear aplicaciones web optimizadas para dispositivos móviles: React se puede utilizar para crear aplicaciones web optimizadas para dispositivos móviles, que se ven y funcionan bien en teléfonos inteligentes y tabletas.
    \item Crear aplicaciones web de una sola página: React se puede utilizar para crear aplicaciones web de una sola página, que no requieren recargas de página para actualizar el contenido.
\end{itemize}

React tiene una serie de características que lo hacen una biblioteca de JavaScript potente y versátil. Estas características incluyen:

\begin{itemize}
    \item Componentes: React se basa en componentes, que son bloques de construcción reutilizables que se pueden combinar para crear interfaces de usuario complejas.
    \item DOM virtual: React utiliza un DOM virtual para representar el estado de una interfaz de usuario. Esto permite que React sea eficiente y reactivo.
    \item Hooks: Los hooks son funciones que permiten a los desarrolladores acceder a los estados y propiedades de los componentes sin tener que escribir clases.
\end{itemize}

\subsubsection*{Ventajas de React}

React ofrece una serie de ventajas sobre otras bibliotecas de JavaScript para crear interfaces de usuario. Estas ventajas incluyen:
\begin{itemize}
    \item Facilidad de uso: React es relativamente fácil de aprender y usar, incluso para desarrolladores principiantes.
    \item Eficiencia: React es una biblioteca eficiente que puede manejar grandes cantidades de datos y eventos.
    \item Reactividad: React es una biblioteca reactiva que puede actualizar las interfaces de usuario en respuesta a los cambios de datos.
    \item Escalabilidad: React es una biblioteca escalable que puede manejar aplicaciones web complejas.
\end{itemize}