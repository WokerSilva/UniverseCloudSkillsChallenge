\section{Inicio de un proyecto}

\begin{enumerate}
    \item Estando dentro de la carpeta Cats:
    \begin{verbatim}
        touch index.html
    \end{verbatim}
    Creación de un archivo
    \item Usar \myBox{sun}{git status} para ver el estado del árbol de trabajo
    \item Ahora usar 
    \begin{verbatim}
        git add .
    \end{verbatim}
    para agregar el nuevo archivo al índice de Git, el punto al final es para agregar todo lo que no tiene seguimiento
    \item Volver a usar un \myBox{sun}{git status}
    \item Realización de la primera confirmación
    \begin{itemize}
        \item Utilizar el comando siguiente para crear otra confirmación:
        \begin{verbatim}
            git commit -m "Añadiendo archivo index.html"
        \end{verbatim}
        \item \myBox{sun}{git status} Para ver que todo salió bien
        \item \myBox{sun}{git log} Para mostrar la información del commit
    \end{itemize}
    \item Modifique index.html y confirme el cambio.
    \item Usando \myBox{Cyan}{code index.html} se abre en VSC el archivo, así mismo si se usa \myBox{Cyan}{code .} en una carpeta se abre toda la carpeta
    \item Modificamos el archivo
    \item Ahora usamos el comando:
    \begin{verbatim}
        git commit -a -m "Add a heading to index.html"
    \end{verbatim}
    \begin{itemize}
        \item No se ha usado \myBox{sun}{git add}
        \item Usamos la marca \myBox{sun}{-a} para agregar los archivos modificados desde la última confirmación. No nuevos
    \end{itemize}
    \item Ahora volvemos a modificar el html
    \item Usando el comando \myBox{sun}{git diff} para ver lo que ha cambiado
\end{enumerate}